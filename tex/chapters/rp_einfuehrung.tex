\chapter{Einführung in Reactive Programming mit RxJava}\label{rp_einfuehrung}
Beschreibung wieso RxJava. Beschreibung was beschrieben wird.
\section{Wie funktioniert Reactive Programming}
Einleitung zum Aufbau: Klassenübersicht des Frameworks mit Erklärung. 
\subsection{Syncronität}
Sync vs. Async - was bringt RP in dieser Hinsicht
\subsection{Parallelisierung}
Concurrency vs. Parallelism - was tritt wie wann auf bzw. kann wie wann angewandt werden
\subsection{Push vs. Pull}
Wichtigster Unteschied. Observable als Gegenpart zu Interable - somit Push vs. Pull Vergleich.
\section{Rx.Observable}
Interface Übersicht. Nutzen und Anwendung anhand von Beispiel. Hot vs. Cold
\section{Rx.Observer}
Was kann Observer -> Interface Übersicht
\subsection{Rx.Subscriber}
Was ist speziell am Subscriber -> Interface Übersicht
\section{Operationen und Transformationen}
Erläuterung von den Stadien der Operation von Beginn über Mitte bis Ende.
\subsection{Exkursion: Streams API Java 8}
Beschreibung was Streams darstellen, wie sich Observables im Vergleich verhalten
\subsection{Operation filter()}
Beispiel und Perlenbild. Einsatz beschreiben
\subsection{Transformation map()}
Beispiel und Perlenbild. Einsatz beschreiben
\subsection{Transformation flatMap()}
Beispiel und Perlenbild. Einsatz beschreiben
\subsection{Operation merge()}
Beispiel und Perlenbild. Einsatz beschreiben
\subsection{Operation zip()}
Beispiel und Perlenbild. Einsatz beschreiben
Eventuell noch mehr Operationen

