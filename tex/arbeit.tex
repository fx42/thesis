%% -----------------------------------------------------------------------------------------------------------------------
%% Diplomarbeit mit LaTeX
%% -----------------------------------------------------------------------------------------------------------------------
\documentclass[a4paper, 11pt, twoside, openright, DIV15, BCOR15mm]{scrbook}
\KOMAoptions{cleardoublepage=empty}
%
% Packages
\usepackage{hanser}
\usepackage{textcomp}
\usepackage{listings}
% Hyperref-Optionen f�r PDF-Files
\usepackage{hyperref}
%% Verhindert Schusterjungen und Hurenkinder
\clubpenalty = 10000
\widowpenalty = 10000 \displaywidowpenalty = 10000
%
% Kein Einzug beim Paragraphenanfang
\parindent0.0cm
\parskip1.5ex
% Stil f�r das Literaturverzeichnis
\bibliographystyle{geralpha}
% Tiefe f�r das Inhaltsverzeichnis
\setcounter{secnumdepth}{2}
\setcounter{tocdepth}{2}
% listings-Package
%+++++++++++++++++++++++++++++++++++++++++++++++++++++++++++++++++
%   Hintergrundfarbe von Quellcode
\definecolor{codecolor}{rgb}{0.85,0.85,0.85}
\lstloadlanguages{[ANSI]C++, Java}
\lstset{basicstyle = \ttfamily \small}
\lstset{backgroundcolor=\color{codecolor}}
\lstset{extendedchars=true} \lstset{showstringspaces = false}
% ++++++++++++++++++++++++++++++++++++++++++++++++++++++++++++++++
%
\raggedbottom
\setlength{\parskip}{2.0ex}
\setlength{\parindent}{0.0cm}
%
% Kopfzeile
\pagestyle{fancy}
\renewcommand{\chaptermark}[1]{\markboth{#1}{}}
\renewcommand{\sectionmark}[1]{\markright{\thesection\ #1}}
\fancyhf{} \fancyhead[L, RO]{\small \thepage}
\fancyhead[LO]{\small \nouppercase  \leftmark}
\fancyhead[RE]{\small \nouppercase \rightmark}
\fancypagestyle{plain}{%
   \fancyhead{} %
   \renewcommand{\headrulewidth}{0pt} %
}
%++++++++++++++++++++++++++++++++++++++++++++++++++++++++++++++++++++
% Abst�nde zwischen Caption und Bild/Tabelle
\setlength\abovecaptionskip          {0.4em}
\setlength\belowcaptionskip          {0.2em}
%++++++++++++++++++++++++++++++++++++++++++++++++++++++++++++++++++++
\hypersetup{
pdftitle = {Beispiel einer Diplomarbeit},
pdfauthor = {},
pdfsubject = {Bachelorarbeit, Masterarbeit},
pdfkeywords = {Bachelorarbeit, Masterarbeit, Fachhochschule Kaiserslautern},
pdfdisplaydoctitle = true,
pdfpagelayout = {SinglePage}
}
% Anteil der Grafiken h�her auf jeder Seite!
\renewcommand{\floatpagefraction}{0.99}
% Kein Auseinanderziehen gegen das Seitenende
\raggedbottom
%
% Beginn Dokument
%
\begin{document}
\thispagestyle{empty}
\title{Beispiel f�r eine Abschlussarbeit}
\date{10.~M�rz~2010}
\author{Manfred Brill}
\maketitle
% Inhaltsverzeichnis
% Schalten Sie auf kleine r�mische Zahlen als Nummerierung um -- sonst
% f�ngt im Inhaltsverzeichnis das erste Kapitel auf Seite 6 an; Seite 1 w�re besser, oder?
\pagenumbering{roman}
\tableofcontents
% Jetzt schalten wir wieder auf arabische Zahlen, und die Z�hler werden zur�ckgesetzt
\pagenumbering{arabic}
% Text
% Verwenden Sie f�r jedes Kapitel eine eigene Datei. Das macht das Schreiben
% leichter!
% Einleitung
\chapter{Einleitung}\label{einleitung}
Ein Beispiel f�r die Hauptdatei einer Diplomarbeit und die Aufteilung der einzelnen
Kapitel in einzelne Dateien, die mit \lstinline$\input$ in die Hauptdatei
\lstinline$beispiel.tex$ integriert werden.

Als Literaturdatenbank wird die Datei \lstinline$arbeit.bib$ verwendet. Die Abbildung
ist sowohl im PNG- als auch im EPS-Format vorhanden. Das Beispiel ist also sowohl
mit \lstinline$latex$ $\to$ \lstinline$dvips$ $\to$ \lstinline$ps2pdf$ als auch direkt
mit \lstinline$pdflatex$ kompilierbar.

% Kapitel 1
%\chapter*{Ehrenw\"ortliche Erkl\"arung}
Hiermit erkl\"are ich, \textbf{Felix} \textbf{Scheidel}, geboren am
\textbf{[Geburts-Datum und -Ort]}, ehrenw\"ortlich,
\begin{itemize}
\item dass ich meine Bachelorarbeit mit dem Titel:

\textbf{Reactive Programming - Ein neues Programmierparadigma}

selbstst\"andig und ohne fremde Hilfe angefertigt habe und keine anderen als in der
Abhandlung angegebenen Hilfen benutzt habe.
\item Die \"Ubernahme w\"ortlicher Zitate aus der Literatur
sowie die Verwendung der Gedanken anderer Autoren an den entsprechenden
Stellen innerhalb der Arbeit gekennzeichnet habe.
\end{itemize}
Ich bin mir bewusst, dass eine falsche Erkl\"arung rechtliche Folgen haben kann.
\vspace*{3cm}

\textbf{Zweibr\"ucken, \today}\hfill\textbf{Felix Scheidel}

% Kapitel 2
%\input{GLSL}
%
% Anhang
%
\appendix
% Literatur
\chaptermark{Literaturverzeichnis}
\sectionmark{Literaturverzeichnis}
\addcontentsline{toc}{chapter}{Literaturverzeichnis}
% Verwenden Sie bibtex!
\bibliography{diplomarbeit}
%
%\chapter*{Ehrenw\"ortliche Erkl\"arung}
Hiermit erkl\"are ich, \textbf{Felix} \textbf{Scheidel}, geboren am
\textbf{[Geburts-Datum und -Ort]}, ehrenw\"ortlich,
\begin{itemize}
\item dass ich meine Bachelorarbeit mit dem Titel:

\textbf{Reactive Programming - Ein neues Programmierparadigma}

selbstst\"andig und ohne fremde Hilfe angefertigt habe und keine anderen als in der
Abhandlung angegebenen Hilfen benutzt habe.
\item Die \"Ubernahme w\"ortlicher Zitate aus der Literatur
sowie die Verwendung der Gedanken anderer Autoren an den entsprechenden
Stellen innerhalb der Arbeit gekennzeichnet habe.
\end{itemize}
Ich bin mir bewusst, dass eine falsche Erkl\"arung rechtliche Folgen haben kann.
\vspace*{3cm}

\textbf{Zweibr\"ucken, \today}\hfill\textbf{Felix Scheidel}

%
%\chapter*{Sperrvermerk}
Die vorliegende Bachelorarbeit/Masterarbeit
\textbf{[Titel der Arbeit]}
enth�lt zum Teil Informationen, die nicht f�r die �ffentlichkeit bestimmt
sind.
Der Inhalt darf daher nur mit der ausdr�cklichen schriftlichen Genehmigung
des Verfassers und \textbf{[Firmenangabe]} an Dritte weitergegeben werden.

Die Arbeit ist nur den Korrektoren sowie erforderlichenfalls den Mitgliedern des
Pr�fungsausschusses zug�nglich zu machen.
\vspace*{3cm}

\textbf{[Ausstellungsort], [Ausstellungsdatum]}\hfill\textbf{[Unterschriften]}

\end{document}
