\pagenumbering{arabic}
\chapter{Einleitung}
\label{cha:einleitung}

Durch den starken Wachstum der IT entstehen immer wieder neue Möglichkeiten Anwendungen zu realisieren. Durch den technischen Fortschritt werden schon bekannte Muster und Architekturen weiter entwickelt oder Ideen für die neu entstandenen Anforderungen umgesetzt. Eine dieser Ideen ist \textit{REACTIVE PROGRAMMING}. Vor noch nicht allzu langer Zeit waren Monolithen die auf eigens gehosteten Servern ausgerollt wurden der Stand der Dinge. Durch die mittlerweile entstandene Vielfalt an Endgeräten wie Smartphones, das Deployment in der Cloud oder die Menge an spezialisierten Programmiersprachen, Frameworks und Entwicklungswerkzeugen haben sich Anforderungen herausgestellt die sich mit bekannten Lösungen wie zum Beispiel einer objektorientierten Herangehensweise nicht zur vollen Zufriedenheit erfüllen lassen. Für einen Benutzer ist es üblich, dass Änderungen sofort sichtbar sind und angefragte Daten in Bruchteilen einer Sekunde bereit stehen, und das zu jedem Zeitpunkt. Kann eine Applikation dies nicht leisten, kann die User Expierience \footnote{TODO: User Experience Def} in Mitleidenschaft gezogen werden, was bei der Menge an Diensten gleicher Art dazu führen kann, dass Benutzer auf die Dienste von Mitbewerber zurück greifen. Um diesem vorzubeugen wurde aus vorhandenen Konzepten ein grundlegendes Manifest für \textit{Reaktive Systeme}\footnote{\cite{Boner.2014}: Reactive Manifesto 2.0. www.reactivemanifesto.org} erstellt. Die im Manifest angeführten Eigenschaften werden im Verlauf dieser Arbeit noch genauer aufgegriffen. Es sei nur jetzt schon gesagt, dass zur Umsetzung der Kriterien für ein Reaktives System, die Verwendung von Reactive Programming nicht notwendig ist, jedoch meist sinnvoll scheint. \\
Ziel dieser Arbeit ist es, eine Einführung in die Welt von Reactive Programming zu geben. Dazu wird im nächsten Kapitel eine Einordnung in das Gesamtbild der Softwareentwicklung durchgeführt. Ebenso werden die Eigenschaften und Eigenheiten von Reactive Programming beschrieben. Weiterhin gibt es einen Überblick über einige Frameworks mit deren Hilfe ein reaktives Programmieren realisiert werden kann. Darauf folgend wird eine Implementierung einer Beispielanwendung besprochen, um dem Vorangegangen eine praktische Anwendung hinzuzufügen.

