\chapter{Was ist Reactive Programming}\label{was_ist_reactive_programming}
Ist mein neu in der Domäne von reaktiver Entwicklung stellt man schnell fest, dass allerhand mögliche Definitionen und Beschreibungen findet was Reactive Programming denn zu sein scheint. Bevor jedoch hier eine Definition erläutert wird, muss zwischen unterschiedlichen Begrifflichkeiten differenziert werden: \textit{Reactive Systems, Functional Reactive Programming} und natürlich \textit{Reactive Programming}. 
\section{Was bedeutet \textit{"reactive"} im Kontext der Softwareentwicklung}
Wie die Wortherkunft ja schon verlauten lässt, wird etwas als reaktiv bezeichnet, wenn eine Reaktion durch eine vorangegangene Aktion ausgelöst wird. Diese Reaktion kann über unterschiedliche Wege in Gang gebracht werden. Grundlegend für diese Interaktion ist die Beobachtung von Daten oder Ereignissen welche von außerhalb der Anwendung, zum Beispiel über die Benutzeroberfläche, oder innerhalb der Anwendung verändert oder ausgelöst werden. Möglichkeiten solch 
\subsection{Differenzierung zwischen Reactive Proramming und Reactive Systems}
Das Manifest bezieht sich eigentlich auf die Systeme. Wo kommt nun RP in Spiel? 
\section{Reactive Programming vs. Functional Reactive Programming}
FRP findet nun mal in Funktionalen Programmiersprachen statt. Java is jedoch OO und mit Java 8 und dem Rx Frameworks werden funktionale Eigenschaften in der OO Sprache eingebracht. Unterschiede müssen noch genau belegt\footnote{\cite{Lohmuller.2016}} werden.
Der grundlegende Gedanke reaktiver Systeme wurde schon im Jahre 1985 in einem Paper von D. Harel und A. Pnueli beschrieben\footnote{\cite{Harel1985}}. 
\section{Reactive Programming - Ein neues Programmierparadigma?}
Wie gliedert man einen Programmierstil ein? \footnote{\cite{Bainomugisha.2013}}
\subsection{Was versteht man unter Programmierparadigmen}
 Erklärung was sind Paradigmen und wieso werden sie definiert.
\subsection{Vergleich: Reactive Programming und Objekt orientierte Programmierung}
Eventuell bessere mit Funktionaler Programmierung zu vergleichen. Muss noch genauer betrachtet werden. Soll die Unterschiede zu den gängigen, bekannten Methoden aufzeigen.
\section{Überblick über bekannte Frameworks und ihre Eigenschaften}
Überblick quer über die gängigen Programmiersprachen. 
\subsection{Reactivex.io}
Kurze Erläuterung zu der Entstehung von Reactive Extensions
\subsection{Allgemeine Übersicht}
Rx Frameworks zu den jeweiligen Sprachen. Frameworks wie z.B. Akka\footnote{\cite{Karnok.2016}}. 
\subsection{Übersicht spezielle für die Entwicklung mit Java}
RxJava. Reactive Streams Konvention. Java 9 Api Änderung bzgl. Reactive Streams.
\subsubsection{Framework für JavaFX - RxJavaFX}
Einführung und Eigenschaften erläutern
\section{Testen von reaktivem Code mit dem JUnit Framework}
Noch nichts genaues. Muss noch geschaut werden wie die Funktionalität von JUnit RP abdeckt.