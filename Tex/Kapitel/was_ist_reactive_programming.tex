\chapter{Was ist Reactive Programming}\label{was_ist_reactive_programming}
Historische Einführung. Wann und von wem wurde das erste Mal von RP gesprochen, wie war die weitere Entwicklung?
Kapitelbeschreibung: Was folgt. 
\section{Reactive Programming - Ein neues Programmierparadigma?}
Wie gliedert man einen Programmierstil ein? \footnote{\cite{Bainomugisha.2013}}
\subsection{Was versteht man unter Programmierparadigmen}
 Erklärung was sind Paradigmen und wieso werden sie definiert.
\subsection{Vergleich: Reactive Programming und Objekt orientierte Programmierung}
Eventuell bessere mit Funktionaler Programmierung zu vergleichen. Muss noch genauer betrachtet werden. Soll die Unterschiede zu den gängigen, bekannten Methoden aufzeigen.
\section{Für was steht das Reactive im Kontext der Programmierung}
Was bedeutet der Begriff Reactive im eigentlichen Sinne, wie genau findet man dass in der Programmierung wieder?
\subsection{Differenzierung zwischen Reactive Proramming und Reactive Systems}
Das Manifest bezieht sich eigentlich auf die Systeme. Wo kommt nun RP in Spiel? 
\section{Reactive Programming vs. Functional Reactive Programming}
FRP findet nun mal in Funktionalen Programmiersprachen statt. Java is jedoch OO und mit Java 8 und dem Rx Frameworks werden funktionale Eigenschaften in der OO Sprache eingebracht. Unterschiede müssen noch genau belegt\footnote{\cite{JanCarstenLohmuller.2016}} werden.
\section{Überblick über bekannte Frameworks und ihre Eigenschaften}
Überblick quer über die gängigen Programmiersprachen. 
\subsection{Reactivex.io}
Kurze Erläuterung zu der Entstehung von Reactive Extensions
\subsection{Allgemeine Übersicht}
Rx Frameworks zu den jeweiligen Sprachen. Frameworks wie z.B. Akka\footnote{\cite{DavidKarnok.2016}}. 
\subsection{Übersicht spezielle für die Entwicklung mit Java}
RxJava. Reactive Streams Konvention. Java 9 Api Änderung bzgl. Reactive Streams.
\subsubsection{Framework für JavaFX - RxJavaFX}
Einführung und Eigenschaften erläutern
\section{Testen von reaktivem Code mit dem JUnit Framework}
Noch nichts genaues. Muss noch geschaut werden wie die Funktionalität von JUnit RP abdeckt.