%----------------- KONFIGURATION ----------------- %
\pagestyle{empty} % enthalten keinerlei Kopf oder Fuß

\section*{Zusammenfassung} % (fold)
\label{cha:zusammenfassung}
In dieser wissenschaftlichen Arbeit wird das Programmierparadigma \textit{Reactive Programming} behandelt. Reactive Programming verbindet die Eigenschafen für asychonres, nicht blockierendes Verarbeiten von Datenströmen. Zu Beginn wird geklärt wie dieses Paradigma in den Kontext der Softwareentwicklung eingeordnet werden kann und wie es sich von Reactive Systems unterscheidet. Anschließend beschäftigt sich diese Arbeit mit den Bestandteilen wie dem Observer Pattern und dem Back Pressure, die dieses asynchrone und nicht blockierende Verhalten generieren. Zusammen bilden sie die Reactive Streams welche ebenfalls genauer erläutern werden. Auch wird eine bereits vorhandene Implementierung in Form der RxJava2 Bibliothek im Detail betrachtet. Es werden Eigenschaften wie Synchronität oder Nebenläufigkeit beschrieben. Weiterhin wird auf die grundlegenden Klassen wie Observables oder Schedulers genauer eingegangen und deren Beschreibungen werden durch Codebeispiele unterstützt. Es folgt eine Implementierung eines Systemmonitors als Beispielapplikation welche die RxJava2 Bibliothek verwendet und einige vorab besprochene Methoden genauer bahandelt. Abschließend wird evaluiert wo Reactive Programming seine Vorteile hat und in welchen Anwendungsbereichen sich dieses Paradigma etablieren kann. Diese Einschätzung legt nah, dass die wahrscheinlich sinnvollsten Gebiete die Arbeit mit REST-Schnittstellen, Microservice-Architekturen und sehr I/O basierte Anwendungen sein werden.


\vspace*{1,5cm}
\section*{Abstract} % (fold)
\label{cha:abtract}
This scientfic work is treating the programming paradigm \textit{Reactive Programming}. This paradigm combines the attributes for asynchronous and non-blocking processing of datastreams. At first Reacitve Programming is classified into the context of software development and the difference to Reactive Systems is exposed. Afterwards this work explains different parts like Observer Pattern or Back Pressure which are necessary to create the asynchronous and non-blocking behavior. Theses parts put together the Reactive Streams, which also get explained in this work. Furthermore an implementation of the already existing library RxJava2 gets discussed in detail. Characteristics like synchronicity and concurrency get described. There will be a closer look into the base classes like Observables and Schedulers accompanied with some coding examples. Ensuing there is an implementation of a system monitoring tool using the RxJava2 library and showing of some of the preceding methods. Finally an evaluation of the benefits of the Reactive Programming paradigm takes place, explaining which area of application this paradigm has its best use. The resulting estimation sees REST-Interfaces, the Microservice Architecture and very I/O heavy applications as the most meaningful areas of effect for Reactive Programming.
 

