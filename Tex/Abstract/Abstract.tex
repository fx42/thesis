%----------------- KONFIGURATION ----------------- %
\pagestyle{empty} % enthalten keinerlei Kopf oder Fuß


\section*{Zusammenfassung} % (fold)
\label{cha:zusammenfassung}
Diese Arbeit behandelt das Programmierparadigma Reactive Programming. Dieses Paradigma findet da Verwendung, wo asynchrone, nicht blockierende Verarbeitung von Daten in Form von Strömen stattfindet. Der Begriff wird in den Kontext der Softwareentwicklung eingeordnet und die einzelnen, notwendigen Eigenschaften, über die eine Anwendung verfügen muss um als reaktiv zu gelten, werden im Detail beschrieben. Die aus diesen Eigenschafen resultierenden Reactive Streams werden in der hier verwendeten Refrenzbibliothek RxJava2 implementiert. Die Realisierung der Eingenschaften wird genauer beschrieben. Ebenso wird die Verwendung der Basisklassen von RxJava2 anhand von Codebeispielen dargestellt. Durch eine Implementierung eines Systemmonitors unter Nutzung dieser Bibliothek werden einige Bestandteile wie die Erstellung von Observables und die Weiterverwendung dieser Streams mit JavaFX realisiert. Nach der Betrachtung dieses Paradigma stellt sich heraus, dass die wahrscheinlich sinnvollsten Anwendungsgebiete die Arbeit mit REST-Schnittstellen, Microservice-Architekturen und sehr I/O basierte Anwendungen sein werden. 

\vspace*{3cm}
\section*{Abstract} % (fold)
\label{cha:abtract}
Diese Arbeit behandelt das Programmierparadigma Reactive Programming. Dieses Paradigma finden da Anwendung wo asynchrone, nicht blockierende Verarbeitung von Daten in Form von Strömen stattfindet. Der Begriff wird in den Kontext der Softwareentwicklung eingeordnet und die einzelnen, notwendigen Eigenschaften, über die eine Anwendung verfügen muss um als reaktiv zu gelten, werden im Detail beschrieben. Die aus diesen Eigenschafen resultierenden Reactive Streams werden in der hier verwendeten Refrenzbibliothek RxJava2 implementiert. Die Realisierung der Eingenschaften wird genauer beschrieben. Ebenso wird die Verwendung der Basisklassen von RxJava2 anhand von Codebeispielen dargestellt. Durch eine Implementierung eines Systemmonitors unter Nutzung dieser Bibliothek werden einige Bestandteile wie die Erstellung von Observables und die Weiterverwendung dieser Streams mit JavaFX realisiert. Nach der Betrachtung dieses Paradigma stellt sich heraus, dass die wahrscheinlich sinnvollsten Anwendungsgebiete die Arbeit mit REST-Schnittstellen, Microservice-Architekturen und sehr I/O basierte Anwendungen sein werden. 

